\subsection{Stakeholders}
The official partner involved in this project is The Composites Research Network (CRN) Okanagan node. The CRN is a group of several Canadian universities and industry partners involved in composites manufacturing. The purpose of the CRN is having academic institutions and industry leaders collaborate to find solutions to problems facing the composites manufacturing industry \cite{crn}. The resulting intellectual property produced from this project could impact the members of the CRN outside of the Okanagan node, so they’re considered external stakeholders even though they were not consulted during this project.

The main stakeholders for this project are Project Supervisor Bryn Crawford, Facility Supervisor Dr. Abbas Milani, and Ph.D. student Reza Sourki at UBC Okanagan campus. These main stakeholders work at UBCO’s Composites and Optimization Laboratory doing research and development on polymeric composite materials. The Project Supervisor Bryn Crawford and Reza Sourki were met with throughout both semesters. Meetings consisted mostly of conceptualizing the problem and understanding their expectations in the first semester. Once the problem was defined, meetings transitioned towards providing updates and progress reports.

The main internal stakeholders at the UBC Okanagan Composites and Optimization Laboratory were impacted positively by the results of this project. Their needs for minimizing soak time and accurately predicting the curing cycle of a composite material were met to an acceptable level. The intellectual property created can also impact other laboratories in the CRN doing composites research and development. There could be benefits for these labs if our framework is adjusted to work for predicting the curing cycle on more complex composite materials.

\subsection{Needs}
Through early discussions with the stakeholders and group meetings, the needs for the project were established. It was found that the main need was to accurately predict a composite's internal temperature and minimize the soak time of the curing cycle. Also made clear by the stakeholders was the need to implement an Industry 4.0 framework creating an innovative and robust final design.\\\\
%The main purpose of this project was to satisfy the needs not being met by the solution in UBC Okanagan’s composite and optimization lab. The main problem with the current solution was the soak time could not be accurately estimated. This led to more power being consumed than necessary in the heat chamber and the possibility that composites are of poorer quality. We knew addressing the main needs was key to reducing the total amount of time for curing a composite material. Another important need to satisfy was building a framework capable of estimating soak times across a wide range of composite materials. This framework also needed to be capable of estimating soak times for varying part geometries, materials, and thermal properties. \\\\
The stakeholders expressed an additional need for implementing an Industry 4.0 framework, combining sensors with machine learning algorithms for estimating soak times. It was determined that using machine learning algorithms would provide a more robust solution.  This meant the framework had to include a way for users to easily store and manipulate data. Finally, the idea of designing a process for uploading data to the IBM Cloud service was expressed if it was feasible. This would give the stakeholders a way of accessing data across multiple devices and provide a platform for doing data calculations. Due to the cost of storing extensive amounts of data, it was defined that our implementation should try to limit the amount of data as to minimize the burden of costs on the lab. It was determined that meeting these needs defined by the project stakeholders would satisfy the requirement of producing an Industry 4.0 framework for this project.\\\\
Accurately finding a given part’s internal temperature over time for the purpose of reducing soak time was considered the most important need for this project. The solution had to satisfy the main need using Industry 4.0 principles including data collection, data streaming, machine learning, and cloud services. Lastly, the overall design framework required detailed documentation so it could be easily replicated over a range of composites.

\subsection{Scope and Constraints}%Many constraints were identified during the definition stage of this project. The main purpose for this project was to develop an Industry 4.0 framework for optimizing the composite curing process. The scope for this project was to build this framework using a machine learning algorithm that estimates the soak time of a metal part with simple geometry. 
Many constraints were identified during the definition stage of this project. The scope of the primary deliverable was building a framework for obtaining internal part temperature using a machine learning algorithm for simple metal cylinders and using this data to estimate the soak time. Studying objects with different materials, sizes, and shapes were considered within the project’s scope if the primary deliverable was completed before the project’s deadline. Metal cylinders were proposed by the stakeholders as the limit of our scope for testing and implementing our Industry 4.0 framework because the heat transfer characteristics are easily known whereas for composite parts the characteristics are not always available. So estimating soak times on composites with epoxy exhibit exothermic properties was considered outside the scope of this project in order to simplify the process for developing a framework.
\\\\
With a budget of \$300, the primary cost constraints were first identified for the project. These included temperature-based sensors, data collection devices like Arduinos and Raspberry Pi, and costs of acquiring metal parts. NTC 100k Thermocouples were implemented for sensing air temperature and part temperature in the heat chamber. Also a 16 bit analog-to-digital converter (ADC) microcontroller and Raspberry Pi were used for creating the data acquisition system for storing and streaming real-time data from the curing cycles. It was planned to store data on the IBM Cloud computing service keeping in mind the volume of data would lead to higher monthly costs for storing this data thus data must be minimized.% It was found that the amount of data collected depended on the number of thermocouples, and the amount of tests performed. Therefore it was an objective to minimize the amount of heat chamber tests performed and find the minimum number of thermocouples without jeopardizing the accuracy of the results. Also the configuration of the thermocouples inside the heat chamber had to be considered for ensuring accurate temperature measurements.
\\\\ 
It was found that the accuracy of the thermocouples was critical because the data collected would be used for training and testing our machine learning algorithms to estimate a part’s internal temperature and soak time required. Therefore constraints on the thermocouples and all hardware devices had to be considered such as the latency and signal-to-noise ratio. To satisfy these constraints, it was important to ensure that the measurements obtained by the hardware devices in this project were consistent and repeatable under specific testing conditions.
\clearpage
The amount of tests performed was also a recognized constraint as it placed restrictions on the volume of data collected and time availability. An objective was to determine the amount of tests necessary for obtaining a large enough data set to meet our expectations. It was found that the time consumed for a single test depended on the target temperature, thermal properties, and the size of the metal cylinder.  Due to these constraints we selected an aluminum cylinder with a 3cm radius and cured it at lower temperatures than typically used for composite curing to have more test runs and a larger data set. Another strategy for meeting these time constraints related to test runs was implementing MATLAB heat simulations to generate more data sets for training our machine learning algorithms and testing them.\\\\
There were some risks identified during this project with regards to health and safety. There were potential dangers working with the heat chamber and infrared lamp because of the high temperatures involved. It was necessary to practice the safety precautions laid out by our clients when using the lab equipment. There was minimal technical risk of losing data due to the use of cloud storage and revision control. Finally there were no significant environmental or societal impacts to be considered within the project's scope.
