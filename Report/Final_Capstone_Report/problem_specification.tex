The main objective of our project was creating an Industry 4.0 framework using a machine learning model to optimize the curing process by accurately estimating the soak time required. Specifically for our case, as a proof of concept we acquired an aluminum cylinder with a radius of 3cm and length of 5cm. The purpose of this proof of concept was to address the problems with the current solution in the UBC Okanagan Composites lab. It was expected that the new framework would implement Industry 4.0 principles integrating data acquisition devices with machine learning. Our clients also expressed a desire for the framework to have the sensor data and the machine learning model uploaded online to the IBM Cloud service. This desire was however dependent on time and costs because the IBM Cloud adds costs based on data usage. Therefore it was clearly stated that the amount of data should be limited as much as possible. However, building a robust machine learning model often requires large amounts of data for training and testing. For addressing the issue, we did not specify a maximum numerical value for the amount of data collected, instead we looked to reduce the amount of data used for training and testing the model until its performance suffered significantly. Using this strategy would be key to producing a framework to satisfy the problem.\\\\
The temperature range to be used for curing parts was specified by the client early in the project. The allowable range was between 25$^\circ$C and 80$^\circ$C for curing parts. Therefore any material placed inside the heat chamber had to be capable of withstanding temperatures of at least 80$^\circ$C. However, our clients did have aspirations for using the final product to test composites at temperatures over 200$^\circ$C. Therefore the specification for our final solution included handling temperatures up to 250$^\circ$C which was the maximum temperature the thermistors could withstand. 
As the project budget was limited to \$300, it was necessary to minimize the cost of hardware devices used for our final design. To stay within budget, we limited our costs on hardware down to devices which included a 16 bit ADC for data acquisition, and a Raspberry Pi for storing the data from the ADC and extracting it for analysis. Ultimately by limiting the costs of hardware devices, we were able to stay within the \$300 budget.