Through this Capstone project, we were able to collaborate with the UBC Okanagan Composites and Optimization laboratory to create an Industry 4.0 solution to the composites manufacturing process in the lab. The objective was to develop a proof of concept machine learning process to optimize the curing cycle of composite materials by estimating the soak time. To develop our framework, an aluminum cylinder was used for testing and gathering data for the machine learning algorithm. Through our project, we were able to improve the data acquisition process by incorporating a Raspberry Pi and ADC to the test configuration. These devices were used to make the curing process simpler and to retrieve more accurate readings from the temperature sensors. From the literature review and research performed, we decided on using Random Forests and LSTM for our machine learning process in order to compare the results given by different methods and to observe which is more effective for our needs. Python was decided for implementing the machine learning algorithms for its extensive selection of machine learning libraries and its accessibility. \\\\
An engineering approach was adopted to resolve concerns that arose during the development of this project. To address the concern of noise, we experimented with different filters and analyzed their effectiveness to improve the signal-to-noise ratio while remaining accurate and seeing how it affected the results of the machine learning processes. Ultimately, we found that noise had little effect on the machine learning algorithms or on the soak time of a test run. Simulations were also performed in MATLAB to complement the real data that was obtained. The simulations allowed us to further understand the test process and allowed us to generate a greater data set to help the modeling of the machine learning algorithms. We developed two machine learning algorithms one using Random Forests and another using a LSTM model. After extensive development on the machine learning processes, we were able to generate predictions with a percent error under 4\% for an entire run. The machine learning algorithms were then applied to predict the soak time of a simple metal part. This could then be used to make a control decision when to stop a curing cycle. The machine leaning algorithm was also used to try and predict how the part would heat up in the future. Despite limitations raised due to the COVID-19 outbreak, we were ultimately able to successfully fulfill the stakeholder's needs. For further development on this design, our recommendations include: accumulation of more data for further fine-tuning of the machine learning processes, more advanced simulations, data assimilation, and transfer learning. 