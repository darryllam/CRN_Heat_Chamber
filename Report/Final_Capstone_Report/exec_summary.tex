At UBC Okanagan’s School of Engineering, fourth year engineering students are required to complete the two semester Capstone design project. Over these two semesters, students of all engineering disciplines work in teams to solve a specific design problem for a client. For this  project, our client was the UBCO Composites Laboratory with the project supervisor Bryn Crawford. This report’s purpose is to detail the project’s final design and the results achieved over the 2019/2020 semesters.  The final report contains the project's needs and constraints, problem specification, design process, solution generation, final design details, and the final design’s evaluation. The project’s fundamental design problem is finding a solution for improving the manufacturing process of composite materials by specifically optimizing the curing cycle of composite materials. The goal was to replace the controller for the composite curing oven which is a standard PID controller with a smart solution implementing Industry 4.0 concepts. Industry 4.0 is the new paradigm digitizing and automating the industrial sector by using digitally connected sensors, machine learning, and real-time data usage. To complete this task, the needs were catalogued. The main needs were established to be finding the soak/cure time through the usage of machine learning and implementing a data collection unit to attain data. Based on this need of finding cure time, a solution was generated and selected by doing an extensive literature review, analysing the key findings, and creating a feasible solution. \\\\
Two machine learning implementations were selected for the preliminary design. The first was random forest which uses decision trees to make predictions on data. The second was long short term memory (LSTM) which is based on a recurrent neural network and is a more traditional neural network machine learning algorithm. To gather data, a Raspberry Pi and ADC were used together to read NTC thermistors inside of the heating chamber. A modified part which allowed for a thermistor to be inserted provided truth data which was used to train the neural network. Air temperature data inside the heat chamber over a cure cycle was also collected which became the input data to be used to make a prediction with the machine learning algorithms. In addition to real data, simulations were used to develop additional data to help iterate and tune our designs. After turning the machine learning algorithms and setting the necessary parameters, a prediction capability with a percent error of under 4\% for the entire run was found. Due to the COVID-19 virus suspending laboratory times, we encountered further constraints that limited testing and the amount of data that could be collected in the lab. Code was then created to estimate the soak time, make future predictions on data, and return control decisions based on machine learning predictions. It was concluded that our implementation provided a good estimate  of the internal temperature of the part . Finally, our recommendations for future topics which could be researched to improve or expand the project include additional validation and testing, implementation with a cloud service for data storage, data assimilation, transfer learning, and improved simulations. 
