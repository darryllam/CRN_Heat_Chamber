The UBCO Composites and Optimization Lab is interested in modernizing the production and fabrication of fibrous and polymeric composites, known as composites manufacturing. The modernization of these processes using the technologies developed over the past two decades is encompassed by the manufacturing paradigm of Industry 4.0. By focusing on the specific process of composite curing, a framework for applying Industry 4.0 principles to composite manufacturing processes can be developed. \\\\
Composite curing is the process of heating a composite material until it reaches a specific internal temperature, in which it undergoes a chemical strengthening process. This temperature will be reached at a specific time, the soak time, which is determined, but not easily calculable, by the initial conditions and physical parameters of the material. This entails that an object heated for its soak time will be cured. For a manufacturing process, the materials are heated in a curing oven, which on a large scale can be expensive to run. In a lab environment, where the soak time of a given material may not be known, curing ovens are left to run much longer than what could reasonably be deemed necessary to compensate for this uncertainty. To minimize the expense, and thus the time the oven is left running for, a new method to determine the soak time for a given material needs to be found. \\\\
Industry 4.0 is a modern paradigm in manufacturing processes whose principles include inter-connectivity, information transparency and decentralized decisions. In practice this amounts to using internet-connected technology to make data-driven decisions for the manufacturing process. With regards to composite manufacturing, the technological framework to accomplish this would include digitally-connected sensors, a centralized database, machine learning, and microcontrollers. For the specific process of composite curing, the sensors would detect temperature and relay this information back to the processing unit, which could then control the curing oven accordingly. By establishing a solution under Industry 4.0 principles, a transferable workflow can be established forming a basis for future research. \\\\
The objective of our project is then to apply the principles of Industry 4.0 to optimize the composite manufacturing process of composite curing. This report will detail the problem formulation, design process, the end product and an evaluation of the overall success of the project. \\\\
